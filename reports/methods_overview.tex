\documentclass{article}
\usepackage{amsmath,amsfonts}
\usepackage{dsfont}
\usepackage{siunitx}

\DeclareMathOperator{\diag}{diag}

\begin{document}

\section{Calculation of tuning coefficients}

We began by factorizing the fluorescence on day 7 using principal components
analysis as follows
\begin{equation}
    \overline{\mathbf{F}} = \mathbf{U} \diag{(\mathbf{s})} \mathbf{C}^\top
\end{equation}
where each row of the matrix $\overline{\mathbf{F}}$ is the trial-averaged
Z-scored fluorescence of one neuron from \SI{2}{\s} before the start of the tone
to \SI{11}{\s} after reward delivery, the columns of $\mathbf{C}$ are principal
components, $\mathbf{s}$ is a vector of singular values in order of decreasing
magnitude, and $\mathbf{U}$ is a matrix of normalized weights. The rest of our
analysis is performed in terms of the top 50 components, corresponding to the
first 50 columns of $\mathbf{C}$, and their associated unnormalized weights,
corresponding to the first 50 columns of the matrix $\mathbf{W} = \mathbf{Us}$.
We refer to these truncated matrices as $\mathbf{C}^\prime$ and
$\mathbf{W}^\prime$.

To determine how the structure of the conditioning task might be represented in
the activity of recorded neurons, we calculated the projections of the tone and
reward stages of the trial onto the neural subspace $\mathbf{C}^\prime$ using
regression
\begin{equation}
    \mathbf{b}_{[x]} =
    ( \mathbf{C}^{\prime\top}\mathbf{C}^\prime )^{-1}
    \mathbf{C}^{\prime\top} \mathds{1}_{[x]}
\end{equation}
where $x$ is tone or reward, and $\mathds{1}_{[\text{tone}]}$ and
$\mathds{1}_{[\text{reward}]}$ are indicator functions for the period from the
start of the tone to the start of reward delivery, and for the first
\SI{2.5}{\s} of reward delivery, respectively.

We defined the tuning coefficient $\tau$ as a measure of the preference of a
neuron for each trial component as follows
\begin{equation}
    \tau_{i[x]} = \frac{\mathbf{w}_i^\prime \mathbf{b}_{[x]}}{\lVert
    \mathbf{w}_i^\prime \rVert \lVert \mathbf{b}_{[x]} \rVert}
\end{equation}
where $\mathbf{w}_i^\prime$ is the vector of weights for the $i$th cell, and
$\lVert \cdot \rVert$ is the $\ell^2$ norm. This is equivalent to the cosine of
the angle between $\mathbf{w}_i^\prime$ and $\mathbf{b}_{[x]}$. In some places
a trial-resolved tuning coefficient $\tau_{ij[x]}$ was used, where $i$ is the
cell index and $j$ is the trial index. This was calculated in a similar way,
substituting $\mathbf{v}_{ij}^\prime = \left(\mathbf{C}^{\prime\top}\mathbf{C}^\prime
\right)^{-1} \mathbf{C}^{\prime\top} \mathbf{f}_{ij}$ (where $\mathbf{f}_{ij}$
is the fluorescence of the $i$th cell during the $j$th trial) for
$\mathbf{w}_i^\prime$ in the previous equation.

\section{Data analysis and code availability}

Principal components analysis and tuning coefficient calculation were performed
using Python 3.8 with the following libraries: NumPy, Pandas, Scikit-Learn,
h5py, and SQLAlchemy. Figures were prepared in Python using matplotlib and
seaborn, and in R using ggplot2. Statistical tests were performed in Python
using SciPy and in R. Code to reproduce the analysis and figures is available
at \texttt{https://github.com/nauralcodinglab/interneuron-reward}. (Right now the
code is private, but we can make it public when you publish.)

\end{document}
