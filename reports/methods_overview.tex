\documentclass{article}
\usepackage{amsmath,amsfonts}
\usepackage{dsfont}
\usepackage{siunitx}

\DeclareMathOperator{\diag}{diag}
\DeclareMathOperator{\corr}{corr}

\begin{document}

\section{Calculation of tuning coefficients}

We quantified the tuning of individual neurons to the tone and reward stimuli
delivered in our classical conditioning task using the Spearman non-parametric
correlation $\rho$ (\texttt{scipy.stats.spearmanr}) between the trial-averaged
fluorescence and an indicator function for the relevant stimulus
\begin{equation}
    \rho_{m,n,s}^{(d)} =
    \corr\left[\frac{1}{\lvert \mathcal{T}_{m}^{(d)} \rvert}\sum_{t \in \mathcal{T}_{m}^{(d)}}\mathbf{f}_{m,n,t:t+T_\text{trial}}^{(d)},
    \mathbf{1}_s\right],
\end{equation}
where $t \in \mathcal{T}_{m}^{(d)}$ is the start of the baseline period before a
trial in mouse $m$ on day $d \in \{1, 7\}$,
$\mathbf{f}_{m,n,t:t+T_\text{trial}}^{(d)}$ is the fluorescence trace of neuron
$n$ from mouse $m$ during a single trial, $\mathbf{1}_s$ is an indicator
function for stimulus $s \in \{\text{tone}, \text{reward}\}$, $U_m^{(d)}$ is the
number of trials, and $\rho$ is the Spearman correlation coefficient.  This
cell-resolved tuning analysis was restricted to a $T_\text{trial} = $\SI{13}{\s}
window around each trial, starting with a 2 second baseline period before the
tone. We considered the ``tone'' period to range from the start of the tone to
the start of reward delivery, and the ``reward'' period to be the first 2.5
seconds of reward delivery. We used the change in $\rho$ from day 1 to day 7 as
a cell-resolved measure of changes in tuning over the course of learning.

To summarize learning-associated changes in tuning, we calculated the mean
change in the Spearman correlation for each cell type and trial component (tone
or reward) from day 1 to day 7 as follows
\begin{equation}
    \overline{\Delta\rho}_{s} = \frac{1}{\lvert \mathcal{M} \rvert} \sum_{m \in \mathcal{M}} \frac{1}{N_m} \sum_{n=1}^{N_m}
    \left(\rho_{m,n,s}^{(7)} - \rho_{m,n,s}^{(1)}\right),
\end{equation}
where $\mathcal{M}$ is the set of mice used in the experiment, $N_m$ is the
number of neurons in mouse $m$, and $\rho_{m,n,s}^{(d)}$ is the Spearman
correlation as defined above.

We used a non-parametric approach for statistical tests involving the mean
change in Spearman correlation by scrambling trial times and bootstrapping mice
to construct a null distribution for $\overline{\Delta\rho}_{s}$. Specifically,
we first drew a random sample of $\lvert \mathcal{M} \rvert$ mice from
$\mathcal{M}$ with replacement, then drew a random sample of $\lvert
\mathcal{T}_m^{(d)} \rvert$ trial start times uniformly distributed between 0
and $T_\text{session}^{(d)} - T_\text{trial}$ for each day $d$ and
randomly-selected mouse, and finally used these randomly-selected mice and
scrambled start times to compute the change in tuning
$\overline{\Delta\rho}_{s}$. This process was repeated 1000 times to approximate
the distribution of $\overline{\Delta\rho}_{s}$ under the null hypothesis that
changes in tuning are unrelated to tone and reward delivery. We considered the
observed changes in tuning $\overline{\Delta\rho}_{s}$ to be statistically
significant at the $*$ or $**$ level if they fell into the \SI{5}{\percent} or
\SI{1}{\percent} tails of this distribution, respectively.

\section{Data analysis and code availability}

Tuning coefficient calculation and statistical tests were performed using Python
3.8 with the following libraries: NumPy, Pandas, h5py, and SQLAlchemy. Figures
were prepared in Python using matplotlib and seaborn, and in R using ggplot2.
Code to reproduce the analysis and figures is available at
\texttt{https://github.com/nauralcodinglab/interneuron-reward}. (Right now the
code is private, but we can make it public when you publish.)

\end{document}
