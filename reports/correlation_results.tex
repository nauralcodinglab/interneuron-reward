\documentclass{article}

\usepackage{amsmath,amsfonts}
\usepackage{dsfont}
\usepackage{siunitx}

\DeclareMathOperator{\diag}{diag}
\DeclareMathOperator{\corr}{corr}

\begin{document}

\section{Main text}

To understand how the representations of rewards and reward-predicting cues
change in M1 over the course of learning, we began by analyzing the tuning
properties of individual cells.  An inspection of the trial-averaged
fluorescence of neurons recorded after seven days of conditioning revealed a
wide range of positive and negative responses to tone (Fig. 2 A) and reward
(Fig. 2 B). Quantifying the tuning of each cell's average response to tone and
reward using the non-parametric Spearman correlation $\rho$ (see Methods)
revealed continuous distributions of tone and reward tuning with no clear
boundaries between positively, negatively, and un-tuned cells (Fig. 2 C, D).
Comparing the tuning properties of the same cells recorded on day 1 and day 7,
we observed a wide distribution of learning-associated changes in tuning within
each cell and even within individual mice (Fig.\ 2 E, F, each curve represents
one mouse). In every animal recorded, we were able to identify examples of
neurons with positive, negative, and virtually no change in tuning to tone and
reward. In spite of this high degree of variability, we found several
significant cell-type specific trends. Specifically, PN and VIP neurons
exhibited opposite changes in reward tuning, with the reward Spearman
correlations of PN decreasing on average from day 1 to day 7 ($\overline{\Delta
\rho}_\text{reward} = -0.141$, Monte-Carlo $p<0.01$, see Methods) while those of
VIP cells increased ($\overline{\Delta \rho}_\text{reward} = 0.161$, $p<0.01$);
at the same time, PN exhibited an increase in the tone Spearman correlation
($\overline{\Delta \rho}_\text{tone} = 0.082$, $p<0.01$). We did not observe any
significant changes in the average tuning properties of PV interneurons
($\overline{\Delta \rho}_\text{tone} = -0.049$, $\overline{\Delta
\rho}_\text{reward} = 0.014$, $p > 0.05$ in each case).  Together, these
observations point to a great diversity in the learning-associated changes in
the response properties of M1 neurons both within and across cell-types.

\section{Figure caption}

Learning-associated changes in single-neuron tuning properties in M1. (A, B)
Representative trial-averaged fluorescence traces, colour-coded based on
non-parametric Spearman correlation with tone (A) or reward (B;\ gray traces at
top). Each trace is one neuron recorded on day 7. (C, D) Trial-averaged
fluorescence traces of all cells recorded on day 7, sorted according to
correlation with tone (C) or reward (D). Tuning coefficient refers to the
Spearman correlation coefficient $\rho$. (E, F) Distribution of changes in
Spearman correlation $\Delta \rho$ with tone (E) or reward (F) according to cell
type. Each curve represents a Gaussian kernel density estimate of the
distribution of $\Delta \rho$ in a single mouse. (G) Mean change in Spearman
correlation $\overline{\Delta \rho}$ according to cell type. Null distributions
(gray) estimated by re-sampling mice and shuffling trial times (see description
of Monte-Carlo simulation in Methods).

\end{document}
