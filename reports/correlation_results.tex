\documentclass{article}

\usepackage{amsmath,amsfonts}
\usepackage{dsfont}
\usepackage{siunitx}

\DeclareMathOperator{\diag}{diag}
\DeclareMathOperator{\corr}{corr}

\begin{document}

\section{Main text}

The next question we asked was whether different cell types changed their
response properties following associative learning. To understand how the
representations of reward and reward-predicting cues changed over the course of
learning, we began by analyzing the tuning properties of individual cells.  By
calculating the correlation between each cell's trial-averaged fluorescence and
the timing of the CS or reward (non-parametric Spearman correlation $\rho$, see
Methods) on day 7 of the experiment, we observed a wide range of tuning
coefficients across cells, with a small proportion of cells being strongly
positively or negatively tuned to the CS or reward stimulus (tuning coefficient
near -1 or 1; Figure 2A-D), consistent with our earlier analyses that neurons in
M1 show activity associated with the CS or reward on day 1 of the conditioning
task. We next examined whether the tuning coefficient changed within each cell
type from day 1 to day 7 after associative learning by calculating the changes
in the tuning coefficients for cell within each mouse from day 1 to day 7
(Figure 2E-G).  On average, PN cells showed a significant increase in CS
Spearman correlation ($\overline{\Delta \rho}$ CS = 0.082, Monte-Carlo $p <
0.01$, see Method Details), suggesting an increase in CS tuning from day 1 to
day 7.  For reward tuning, PNs and VIP-INs exhibited opposite changes on
average: the reward Spearman correlations of PNs decreased from day 1 to day 7
($\overline{\Delta \rho}$ reward = -0.141, Monte-Carlo $p < 0.01$), while those
of VIP-INs increased ($\overline{\Delta \rho}$ reward = 0.161, Monte-Carlo $p <
0.01$). We did not observe any significant changes in the average tuning
properties of PV-INs in either case ($\overline{\Delta \rho}$ CS = -0.049,
$\overline{\Delta \rho}$ reward = 0.014, Monte-Carlo $p > 0.05$). Overall, the
single-cell analyses revealed a trend that the response properties may
undergo cell-type specific modifications during associative learning.

Because neuronal population codes can be more or less than the sum of their
parts (Pillow et al., 2008; de Vries et al., 2020; see Averbeck et al., 2006 for
review), we next asked how the responses of populations of each cell type
changed over the course of learning. In particular, we sought to determine
whether populations that exhibited stronger responses to either CS or reward
after learning did so because of either (1) an increase in the number of neurons
being recruited during the stimulus, or (2) an increase in the trial-to-trial
reliability of neurons that already responded to the stimulus on day 1. ...

Pillow, J., Shlens, J., Paninski, L. et al. Spatio-temporal correlations and
visual signalling in a complete neuronal population. Nature 454, 995–999 (2008).
https://doi.org/10.1038/nature07140

de Vries, S.E.J., Lecoq, J.A., Buice, M.A. et al. A large-scale standardized
physiological survey reveals functional organization of the mouse visual cortex.
Nat Neurosci 23, 138–151 (2020). https://doi.org/10.1038/s41593-019-0550-9

Averbeck, B., Latham, P. \& Pouget, A. Neural correlations, population coding and
computation. Nat Rev Neurosci 7, 358–366 (2006). https://doi.org/10.1038/nrn1888

\section{Figure caption}

Learning-associated changes in single-neuron tuning properties in M1. (A, B)
Representative trial-averaged fluorescence traces, colour-coded based on
non-parametric Spearman correlation with tone (A) or reward (B;\ gray traces at
top). Each trace is one neuron recorded on day 7. (C, D) Trial-averaged
fluorescence traces of all cells recorded on day 7, sorted according to
correlation with tone (C) or reward (D). Tuning coefficient refers to the
Spearman correlation coefficient $\rho$. (E, F) Distribution of changes in
Spearman correlation $\Delta \rho$ with tone (E) or reward (F) according to cell
type. Each curve represents the cells from one mouse, with the height of the
curve being proportional to the number of cells (area under the curve normalized
to 1). Distributions were estimated using Gaussian kernel density estimation. PN
$N=6$ mice, $N = 187 - 444$ cells per mouse; PV $N=6$ mice, $N = 41 - 136$ cells
per mouse; VIP $N=5$ mice, $N = 94 - 189$ cells per mouse.  (G) Mean change in
Spearman correlation $\overline{\Delta \rho}$ across all mice and cells
according to cell type. Null distributions (gray) estimated by re-sampling mice
and shuffling trial times (see description of Monte-Carlo simulation in
Methods). Sample sizes same as in E and F.

\end{document}
